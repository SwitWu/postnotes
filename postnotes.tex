% \iffalse meta-comment
%
% File: postnotes.tex
%
% This file is part of the LaTeX package "postnotes".
%
% Copyright (C) 2022  Gustavo Barros
%
% It may be distributed and/or modified under the conditions of the
% LaTeX Project Public License (LPPL), either version 1.3c of this
% license or (at your option) any later version.  The latest version
% of this license is in the file:
%
%    https://www.latex-project.org/lppl.txt
%
% and version 1.3 or later is part of all distributions of LaTeX
% version 2005/12/01 or later.
%
%
% This work is "maintained" (as per LPPL maintenance status) by
% Gustavo Barros.
%
% This work consists of the files postnotes.dtx,
%                                 postnotes.ins,
%                                 postnotes.tex,
%                                 postnotes-code.tex,
%         and the files listed in MANIFEST.md.
%
% The released version of this package is available from CTAN.
%
% -----------------------------------------------------------------------
%
% The development version of the package can be found at
%
%    https://github.com/gusbrs/postnotes
%
% for those people who are interested.
%
% -----------------------------------------------------------------------
%
% \fi

\documentclass{l3doc}

% The package itself *must* be loaded so that \GetFileInfo can pick up date
% and version data.
\usepackage{postnotes}
\postnotesetup{
  heading={
    \section*{\pntitle}
    \markright{\pnheaderdefault}
    \addcontentsline{toc}{section}{\pntitle}
  } ,
}

\usepackage[T1]{fontenc}

\usepackage[sc]{mathpazo}
\linespread{1.05}
\usepackage[scale=.88]{tgheros} % sans
\usepackage[varqu,scaled=1.03]{inconsolata} % tt
\usepackage{microtype}

\hypersetup{hidelinks}

\usepackage{zref-clever}
\zcsetup{
  cap,
  check,
  titleref,
  countertype = { pnexample = example } ,
}

\usepackage{listings}
\lstdefinestyle{code}{
  language=[LaTeX]TeX,
  moretexcs={
    \chapter ,
    \@mkboth ,
    \@textsuperscript ,
    \AddToHook ,
    \counterwithin ,
    \tableofcontents ,
  }
}
\lstdefinestyle{postnotes}{
  style=code,
  moretexcs={
    \postnote ,
    \postnoteref ,
    \postnotesection ,
    \printpostnotes ,
    \pntitle ,
    \pnheaderdefault ,
    \pnthechapter ,
    \pnthesection ,
    \pnthechapternextnote ,
    \pnthesectionnextnote ,
  }
}
\lstset{
  style=postnotes,
  basicstyle=\ttfamily\small,
  columns=fullflexible,
  keepspaces,
  xleftmargin=\leftmargin,
  xrightmargin=.5\leftmargin,
}
% Setup inspired by https://tex.stackexchange.com/a/4068. For how to use these
% environments in a .dtx context see https://tex.stackexchange.com/a/31026.
\newcounter{pnexample}
\lstnewenvironment{pnexample}[1][]{%
  \renewcommand{\lstlistingname}{Example}%
  \renewcommand*\theHlstlisting{ht.\thelstlisting}%
  \lstset{#1}%
  \setcounter{lstlisting}{\value{pnexample}}%
}{}
\lstnewenvironment{pnsnippet}[1][]{%
  \renewcommand{\lstlistingname}{Example}%
  \lstset{#1}%
}{}
\ExplSyntaxOn
\makeatletter
\lst@AddToHook { PreInit }
  {
    \cs_if_exist:cT { c@ \lstenv@name }
      { \exp_args:Nx \refstepcounter { \lstenv@name } }
  }
\makeatother
\ExplSyntaxOff

\NewDocumentCommand\opt{m}{\texttt{#1}}

\begin{document}

\GetFileInfo{postnotes.sty}

\title{%
  The \pkg{postnotes} package%
  \thanks{This file describes \fileversion, released \filedate.}%
  \texorpdfstring{\\{}\medskip{}}{ - }%
  User manual%
  \texorpdfstring{\medskip{}}{}%
}

\author{%
  Gustavo Barros%
  \thanks{\url{https://github.com/gusbrs/postnotes}}%
}

\date{\filedate}

\maketitle

\begin{center}
  {\bfseries \abstractname\vspace{-.5em}\vspace{0pt}}
\end{center}

\begin{quotation}
  \pkg{postnotes} is an endnotes package for \LaTeX{}.  Its user interface
  provides means to print multiple sections of notes along the document, and
  to subdivide them either automatically -- by chapter, by section -- or at
  manually specified places, thus being able to easily handle both numbered
  and unnumbered headings.  The package also provides infrastructure for
  setting up contextual running headers for printed notes.  The default is a
  simple but useful one, in the form ``Notes to pages N--M'', but more
  elaborate ones can be built.  When \pkg{hyperref} is loaded, \pkg{postnotes}
  provides hyperlinked notes, including back links.
\end{quotation}

\clearpage{}

\tableofcontents

\clearpage{}

\section{Introduction}

\pkg{postnotes} is an endnotes package for \LaTeX{}.  Its user interface
provides means to print multiple sections of notes along the document, and to
subdivide them either automatically -- by chapter, by section -- or at
manually specified places, thus being able to easily handle both numbered and
unnumbered headings.  The package also provides infrastructure for setting up
contextual running headers for printed notes.  The default is a simple but
useful one, in the form ``Notes to pages N--M'', but more elaborate ones can
be built.  When \pkg{hyperref} is loaded, \pkg{postnotes} provides hyperlinked
notes, including back links.

Though this feature set is mostly (albeit not completely) available in one or
another of the existing endnotes packages for \LaTeX{}, subsets of it exist in
individual packages, not necessarily compatible with each other.
\pkg{postnotes} brings these features together in one place, with no external
dependencies except an up-to-date kernel.

On the technical side, \pkg{postnotes} is peculiar among existing \LaTeX{}
packages in this area of functionality by the fact that it does not use an
external file to store the notes.  Both the notes' contents and its metadata
are stored in variables which are later retrieved at the time of printing.  In
particular, the content of the note is stored and retrieved with ``no
manipulation'' (as in \texttt{expl3}'s \texttt{N}/\texttt{n} function
signatures) and only gets to be expanded at the time it is meant to be
typeset.  The \file{.aux} file is leveraged to set page labels for the notes,
since that particular information has to be retrieved asynchronously but,
other than that, variables are employed to pass information around.

This has some advantages.  First, as is well known, sending arbitrary content
to a file to be read later is not a noiseless process in \LaTeX{}.  Thus, not
doing so makes things smoother.  Second, the external file approach is
strictly linear: the notes which were written to the file get printed as such,
in the order they were written.  Having the notes available as a set of
variables allows for some more flexibility than that, through the possibility
of pre-processing the notes before printing.  It also brings some extra
degrees of freedom in storing note metadata, and in restoring part of the
environment where the note is called to where the note's content is printed.

\section{Loading the package}

\pkg{postnotes} can be loaded with the usual:

\begin{pnsnippet}
\usepackage{postnotes}
\end{pnsnippet}

The package does not accept load-time options, package options must be set
using \cs{postnotesetup} (see \zcref{sec:options},
\zcref[ref=title,noname]{sec:package-options}).

\section{User interface}

\begin{function}{\postnote}
  \begin{syntax}
    \cs{postnote}\oarg{options}\marg{text}
  \end{syntax}
\end{function}
Sets a postnote with content \meta{text}.  A note ``mark'' is typeset at the
place \cs{postnote} is called, and \meta{text} is stored to be typeset later,
on the next call to \cs{printpostnotes}.  The mark is usually determined by
the printed representation of the main counter, \texttt{postnote}, but can be
manually set too.  \cs{postnote} can receive a number of \meta{options}, which
are presented in \zcref{sec:options},
\zcref[ref=title,noname]{sec:note-options}.

\begin{function}{\postnotesection}
  \begin{syntax}
    \cs{postnotesection}\oarg{options}\marg{text}
  \end{syntax}
\end{function}
Sets a postnote section with content \meta{text}.  This is the basic interface
for making notes subdivisions, and it places \meta{text} between the notes
where it occurs, at the point the notes are printed by \cs{printpostnotes}.
For more details and some examples, see \zcref{sec:notes-sections}.  Its
\meta{options} are presented in \zcref{sec:options},
\zcref[ref=title,noname]{sec:section-options}.

\begin{function}{\printpostnotes}
  \begin{syntax}
    \cs{printpostnotes}
  \end{syntax}
\end{function}
Prints the \cs{postnotes} set since the last call of \cs{printpostnotes}, or
since the beginning of the document.

\begin{function}{\postnoteref}
  \begin{syntax}
    \cs{postnoteref}\meta{*}\marg{label}
  \end{syntax}
\end{function}
Typesets a postnote reference to \meta{label}.  Of course, \meta{label} must
have been set to a particular postnote, which can be done by the standard
\cs{label} command.  The starred version of the command inhibits hyperlinking.
When the \pkg{zref-user} package is loaded, a corresponding \cs{postnotezref}
is also provided.


\section{Options}
\zlabel{sec:options}

\subsection*{Package options}
\zlabel{sec:package-options}

\begin{function}{\postnotesetup}
  \begin{syntax}
    \cs{postnotesetup}\oarg{options}
  \end{syntax}
\end{function}
Package options can be configured by means of \cs{postnotesetup}, which
receives options and values in \texttt{key=value} fashion.

\bigskip{}

\DescribeOption{heading} %
\DescribeOption{\cs{pnheading}} %
The \opt{heading} option sets the heading for the printed notes or, more
generally put, that which is printed at the beginning of \cs{printpostnotes}.
It's default value depends on the document class in use.  If \cs{chapter} is
defined, it's default is:
\begin{pnsnippet}
\chapter*{\pntitle}
\@mkboth{\pnheaderdefault}{\pnheaderdefault}
\end{pnsnippet}
but otherwise, it is:
\begin{pnsnippet}
\section*{\pntitle}
\markright{\pnheaderdefault}
\end{pnsnippet}
where \cs{pntitle} is localizable string, which by default contains ``Notes''
(see \zcref{sec:localization}), and \cs{pnheaderdefault} is a function which
takes no arguments, but processes a number of variables, to set a contextual
running header for the printed notes (see \zcref{sec:headers}).
\cs{pnheaderdefault} produces a header in the form ``Notes to pages N--M'',
according to the notes in each page.  If you prefer, you can redefine
\cs{pnheading} instead of using the \opt{heading} option, to the same effect.

\DescribeOption{format} %
The \opt{format} option stores formatting instructions for printing the
notes.  It is called at \cs{printpostnotes}, every time a block of notes is
about to start.  The default value is \cs{small}.

\DescribeOption{listenv} %
The \opt{listenv} option sets the list environment inside which the notes are
printed in \cs{printpostnotes}.  This must be the name of an existing list
environment, and \pkg{postnotes} provides two for convenience:
\env{postnoteslist}, which is the default, and \env{postnoteslisthang} which
typesets the notes with a hanging indent.  You can also create your own, with
\pkg{enumitem} or otherwise, of course.  Either way, each note in the list is
laid out in the form:
\begin{pnsnippet}[escapeinside=`']
\item[`\meta{mark}']`\meta{note content}'
\end{pnsnippet}
\opt{listenv} can also receive the special value \texttt{none}, in which case
the notes blocks are not wrapped in a list environment, but rather typeset as
plain paragraphs.  In this case, each note is laid out in the form:
\begin{pnsnippet}[escapeinside=`']
`\meta{mark}'`\meta{note content}'\par
\end{pnsnippet}
Note that if you use \texttt{\opt{listenv}=none}, you'll probably also want to
set \opt{maketextmark} to a different value than the default, possibly to add
a space between the mark and the content, or to make the mark a superscript.

\DescribeOption{makemark} %
\DescribeOption{maketextmark} %
\DescribeOption{\cs{pnthepage}} %
The \opt{makemark} and \opt{maketextmark} options control how the mark is to
be typeset, at the point \cs{postnote} is called and at the point the note's
text is printed at \cs{printpostnotes}, respectively.  They both can receive
three arguments: \texttt{\#1} is the mark itself, and arguments \texttt{\#2}
and \texttt{\#3} are, respectively, the start and the end of the backlink
(hence they must be used in this order).  Their default values are:
\begin{pnsnippet}
makemark = {#2\hbox{\@textsuperscript{\normalfont#1}}#3} ,
maketextmark = {#2#1.#3} ,
\end{pnsnippet}
At the point \opt{maketextmark} gets typeset, the \cs{pnthepage} variable
contains the value of \cs{thepage} where its corresponding note was set.

\DescribeOption{hyperref} %
\DescribeOption{backlink} %
The \opt{hyperref} option controls the use of \pkg{hyperref} by
\pkg{postnotes} and takes values \opt{auto}, \opt{true} or \opt{false}.  The
default value, \opt{auto}, makes \pkg{postnotes} use \pkg{hyperref} if it is
loaded.  \opt{true} does the same thing, but warns if \pkg{hyperref} is not
loaded (\pkg{hyperref} is never loaded for you).  \opt{false} means not to use
\pkg{hyperref} regardless of its availability.  The \opt{backlink} option
controls whether only a link from the note to is respective text at
\cs{printpostnotes} is created, or if a backlink from the text at
\cs{printpostnotes} back to where the note's mark is placed is also made
available.  It is a boolean option, and is only operational if \opt{hyperref}
is not \texttt{false}.  These are a preamble only options.

\DescribeOption{sort} %
The \opt{sort} option controls whether the notes queue is sorted or not at
\cs{printpostnotes}.  Normally, the order the notes should be printed is the
one in which the notes were placed along the document.  However, in cases
where some manual intervention was required, sorting the notes can be quite
useful, and difficult to handle in its absence.  Two typical examples are: a
note inside a float which disturbed the sequence of the \texttt{postnote}
counter and a manually set mark, in which case \pkg{postnotes} also allows to
manually set a sort value with the \opt{sortnum} option of \cs{postnote}.
Sorting does not cross boundaries of notes sections, as set by
\cs{postnotesection}, in other words, if notes sections exist, sorting is only
ever carried out within the boundaries of each section.  This may be a
restriction for cases in which floats cross sections' boundaries, but it's the
only reasonable thing to do.  \opt{sort} is a boolean option, and defaults to
\texttt{true}.

To appreciate the role the sorting of notes can have, consider the document in
\zcref{ex:opt:1}, which produces correctly ordered note marks, and correctly
ordered printed notes:

\begin{pnexample}[caption={Sorting and floats},label={ex:opt:1}]
\documentclass{book}
\usepackage{postnotes}
\usepackage{hyperref}
\begin{document}
\chapter{First chapter}
\postnote{1}
\postnote{2}
\begin{table}[p]
  \caption{Table}
  Table.\postnote[mark=5,sortnum=5]{3}
\end{table}
\postnote{4}
\postnote{5}
\stepcounter{postnote}
\clearpage
\postnote{6}
\printpostnotes
\end{document}
\end{pnexample}

\DescribeOption{style} %
\opt{style} is just a convenience ``meta'' option which sets a number of
``base'' options -- such as \opt{listenv}, \opt{format}, \opt{maketextmark},
etc.\ -- in order to emulate known styles of printing the notes.  It accepts
the values \texttt{endnotes} or \texttt{pagenote} so that \cs{printpostnotes}
works as its counterparts in each of these packages.


\subsection*{Note options}
\zlabel{sec:note-options}

The options accepted by \cs{postnote}\oarg{options}\marg{text} are the
following:

\bigskip{}

\DescribeOption{mark} %
By default, the mark generated by \cs{postnote} is determined by the printed
representation of the \texttt{postnote} counter, \cs{thepostnote}, which is
stepped when \cs{postnote} is called.  But the \opt{mark} option allows you to
manually set it, in case you want, or need, to do so.  When \opt{mark} is
manually set, the \texttt{postnote} counter is not stepped.  Note that,
differently from the optional argument of \cs{footnote}, this does not need to
be a number, it can receive some text as value, which is directly used as the
mark.

\DescribeOption{sortnum} %
Normally, the sort value used for sorting the notes queue (see the \opt{sort}
package option above) is determined by the value of the \texttt{postnote}
counter (that is, by \cs{the}\cs{c@postnote}, and not by its printed
representation \cs{thepostnote}).  But you may specify this sort value
manually with the \opt{sortnum} option, typically, when you have also manually
specified the mark.  It receives a floating point number as value.  So, for
example, if one needed to insert a note between notes 2 and 3, without
disturbing the numbering of other notes, one could use
\texttt{\cs{postnote}[mark=2*,sortnum=2.5]\marg{text}}.

\DescribeOption{nomark} %
The \opt{nomark} option makes \cs{postnote} inhibit the typesetting of the
mark.  Of course, normally, we do want the visual cue of the mark, but the
intended use case for this option is for a \cs{postnote} with \opt{nomark} to
be paired with a \cs{postnoteref}, so as to be able to typeset a note in
places where doing so directly may be problematic.  For example, if you set a
\cs{postnote} inside a standard caption, whose text is long enough to require
two lines, the content of the caption ends up being typeset twice: once to
check if it would have fitted in a single line, the second to typeset the two
lines since it didn't fit in one.  This triggers the \texttt{postnote} counter
to be stepped twice.  One way to handle this situation is to use the paring
between a \opt{nomark} \cs{postnote} and \cs{postnoteref}:
\begin{pnsnippet}
\begin{figure}
  Figure.
  \postnote[nomark]{\label{en:1}A note.}%
  \caption[A short caption]{A long caption, enough to require two
    lines\postnoteref{en:1}}
\end{figure}
\end{pnsnippet}

\DescribeOption{label} %
The \opt{label} option sets a standard \cs{label} named with the value given
to the option.  When the \pkg{zref-user} package is loaded, a corresponding
\opt{zlabel} option is also provided.  See \zcref{sec:cross-references} for
details about cross-referencing.


\subsection*{Section options}
\zlabel{sec:section-options}

The options accepted by \cs{postnotesection}\oarg{options}\marg{text} are the
following:

\bigskip{}

\DescribeOption{name} %
For the purposes of building running headers, each \cs{postnotesection} can be
identified by its \opt{name}.  This is mainly intended to support unnumbered
headings, but its mechanism is general.  The name of a note section, if the
option has been set to it, is made available for the first and last note on a
given page through the variables \cs{pnhdnamefirst} and \cs{pnhdnamelast} at
the moment the header of the page is typeset.  For details on how to use these
variables, see \zcref{sec:headers}.


\section{Notes sections}
\zlabel{sec:notes-sections}

As mentioned above, \cs{postnotesection} is the basic interface for
subdividing the notes when printed.  For those familiar with it, this command
is \pkg{postnotes}'s equivalent to \pkg{endnotes}' \cs{addtoendnotes}.  It has
the same intended use -- to add text or commands along the notes' sequence at
the point it is called -- and the way it works is quite similar to
\cs{addtoendnotes}.  But there are some differences, prominently a
\cs{postnotesection} is skipped at \cs{printpostnotes} if it contains no
notes.  In other words, if two (or more) calls of \cs{postnotesection} occur
in immediate sequence, with no \cs{postnote} in between, the latter call takes
precedence over the former, instead of being accumulated in the queue.  This
is intended to facilitate the automation of the subdivision of the notes.  So,
one can, for example, use a hook to \cs{chapter} and not have to worry if a
chapter with no notes will generate an empty section inside
\cs{printpostnotes}, e.g., by the call to \cs{chapter*} at the table of
contents, and so on.  Or, one can use the heading number for the automated
case, but be able to override it manually for an occasional unnumbered one.
For this reason, a more semantic name was chosen for it, instead of the
generic ``add to''.

\DescribeOption{\cs{pnthechapter}} %
\DescribeOption{\cs{pnthesection}} %
\DescribeOption{\cs{pnthechapternextnote}} %
\DescribeOption{\cs{pnthesectionnextnote}} %
Just like with \cs{postnote}, the contents of \cs{postnotesection} are not
expanded in place, but rather stored with ``no manipulation'' to be typeset
later at \cs{printpostnotes}.  For this reason, some contextual information is
stored at the place \cs{postnotesection} is called, and made available at the
point it's content gets expanded by means of some variables.  When the content
of a notes section gets typeset, \cs{pnthechapter} contains the value of
\cs{thechapter} where \cs{postnotesection} was originally called.  Similarly,
\cs{pnthesection} contains the value of \cs{thesection}.
\cs{pnthechapternextnote} and \cs{pnthesectionnextnote} are meant to support
the automation of the notes subdivision by using hooks to sectioning commands,
which is a quite natural way to do this.  However, since it may be problematic
to hook \emph{after} sectioning commands -- \cs{section}, for example, figures
prominently in the documentation of \pkg{ltcmdhooks} as a case of ``look
ahead'' command for which the \texttt{after} hook is not supported -- we will
typically want to hook \emph{before} them.  But, at this point the values of
the \texttt{chapter} or \texttt{section} counters have not yet been stepped,
therefore \cs{thechapter} and \cs{thesection} do not correspond to what we
would like to refer to.  For this reason, \cs{pnthechapternextnote} contains
the value of \cs{thechapter} at the point the ``next note'' is placed (a
\cs{postnote}, that is), the first in chapter, and already inside it, thus
with an expected value of the \texttt{chapter} counter.  Similarly,
\cs{pnthesectionnextnote} contains the value of \cs{thesection} for the ``next
note''.  Of course, the ``next note'' variables are \emph{proxies}, but they
are meant to support the automation of the subdivision of the notes through
the use of \texttt{before} hooks to the document's sectioning commands, in
which case the subdivision of the notes will correspond to the document's
structure and, given empty notes sections are skipped, the values will be the
ones we are interested in.  But more complex use cases may require something
different.  Either way, it is up to the user to judge whether the \emph{proxy}
is a good one for their use case, the variables just do what they say, and
contain the values of interest for the ``next note''.

This is meant to be simple.  Some examples might make things more concrete.
At its basic, \cs{postnotesection} can just be set manually:

\begin{pnexample}[caption={Basic usage},label={ex:sect:1}]
\documentclass{book}
\usepackage{postnotes}
\usepackage{hyperref}
\begin{document}
\chapter{First chapter}
\postnotesection{\section*{Notes to chapter \pnthechapter}}
Foo.\postnote{Foo note.}
Bar.\postnote{Bar note.}
\chapter{Second chapter}
\postnotesection{\section*{Notes to chapter \pnthechapter}}
\setcounter{postnote}{0}
Baz.\postnote{Baz note.}
Boo.\postnote{Boo note.}
\printpostnotes
\end{document}
\end{pnexample}

The document in \zcref{ex:sect:1} resets the \texttt{postnote} counter for each
chapter, and manually sets notes sections by chapter, which results in
\cs{printpostnotes} being correspondingly subdivided.  But it is easy to make
this automatic:

\begin{pnexample}[caption={Automating notes subdivision with a hook},label={ex:sect:2}]
\documentclass{book}
\usepackage{postnotes}
\AddToHook{cmd/chapter/before}{%
  \postnotesection{\section*{Notes to chapter \pnthechapternextnote}}}
\counterwithin*{postnote}{chapter}
\usepackage{hyperref}
\begin{document}
\tableofcontents
\chapter{First chapter}
Foo.\postnote{Foo note.}
Bar.\postnote{Bar note.}
\chapter{Second chapter}
Baz.\postnote{Baz note.}
Boo.\postnote{Boo note.}
\printpostnotes
\end{document}
\end{pnexample}

\zcref{ex:sect:2} uses the \texttt{cmd/chapter/before} hook, and thus
\cs{pnthechapternextnote} to retrieve the correct chapter number for
\cs{postnotesection}, as explained above.  The counter is reset every chapter
by means of \cs{counterwithin*}.  Note that the call to \cs{chapter*} inside
\cs{tableofcontents} does not generate a spurious notes section at
\cs{printpostnotes} (as long as you don't place a note in a sectioning command
with no short argument, which you shouldn't do anyway).  But what if we have,
among mostly numbered chapters, an ocasional unnumbered one?
\cs{pnthechapternextnote} wouldn't possibly work in this case.  Since
immediately successive calls to \cs{postnotesection} override the previous
ones, it is straightforward to just manually adjust the exception:

\begin{pnexample}[caption={Fine-tuning automation},label={ex:sect:3}]
\documentclass{book}
\usepackage{postnotes}
\AddToHook{cmd/chapter/before}{%
  \postnotesection{\section*{Notes to chapter \pnthechapternextnote}}}
\counterwithin*{postnote}{chapter}
\usepackage{hyperref}
\begin{document}
\tableofcontents
\chapter*{Introduction}
\postnotesection{\section*{Notes to the introduction}}
Intro.\postnote{Intro note.}
\chapter{First chapter}
Foo.\postnote{Foo note.}
Bar.\postnote{Bar note.}
\chapter{Second chapter}
Baz.\postnote{Baz note.}
Boo.\postnote{Boo note.}
\printpostnotes
\end{document}
\end{pnexample}

\section{Headers}
\zlabel{sec:headers}

\section{Cross-references}
\zlabel{sec:cross-references}

Cross-referencing with \pkg{postnotes} works in a pretty standard way: set a
label, make references to it.  However, there are two ways to set a label to a
note.  One can either set a label with the \opt{label} option of
\cs{postnote}, or one can directly set it with the standard \cs{label} as part
of the note's content.  They are both valid, but they are not equivalent, they
have different meanings and, accordingly, behave differently.

The label set with the \opt{label} option is set at the place where
\cs{postnote} is.  The label set with \cs{label} in the note's content, is
just stored, and only gets expanded when this content gets to be typeset, at
\cs{printpostnotes}.  In short, the \opt{label} option belongs to the
``mark'', while the \cs{label} set in the content belongs to the ``text''.

Of course, the data stored in each label will correspond to this difference.
Even if the plain \cs{ref} to both will get the same value (the mark), a
\cs{pageref} will be different, the links to either will point to different
places, etc.


\section{Localization}
\zlabel{sec:localization}

\section{Acknowledgments}

Some people have kindly contributed to \pkg{postnotes}, whether they are aware
of it or not.  Suggestions, ideas, solutions to problems, bug reports or even
encouragement were generously provided by (in chronological order):
  Ulrike Fischer,
  % 2022-03-22: https://chat.stackexchange.com/transcript/message/60708390#60708390
  % 2022-03-28: https://chat.stackexchange.com/transcript/message/60754383#60754383
  % 2022-03-31: https://github.com/latex3/hyperref/issues/230
  % 2022-04-09: https://github.com/latex3/hyperref/issues/229
  David Carlisle,
  % 2022-03-28: https://chat.stackexchange.com/transcript/message/60754383#60754383
  % 2022-04-08: https://tex.stackexchange.com/a/640035 (comments)
  and Moritz Wemheuer.
  % 2022-04-05: https://tex.stackexchange.com/q/597359#comment1594585_597389

If I have inadvertently left anyone off the list I apologize, and please let
me know, so that I can correct the oversight.

Thank you all very much!


\section{Change history}

A change log with relevant changes for each version, eventual upgrade
instructions, and upcoming changes, is maintained in the package's repository,
at \url{https://github.com/gusbrs/postnotes/blob/main/CHANGELOG.md}.  The
change log is also distributed with the package's documentation through CTAN
so, most likely, \texttt{texdoc postnotes/changelog} should provide easy local
access to it.  An archive of historical versions of the package is also kept
at \url{https://github.com/gusbrs/postnotes/releases}.

\end{document}
