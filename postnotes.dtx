% \iffalse meta-comment
% \fi
%
% \iffalse
%<*driver>
\documentclass{l3doc}

% Have \GetFileInfo pick up date and version data and used in the
% documentation.
\usepackage{postnotes}

\begin{document}

\DocInput{postnotes.dtx}

\end{document}
%</driver>
%
% \fi
%
%
% \NewDocumentCommand\opt{m}{\texttt{#1}}
%
% \pdfstringdefDisableCommands{^^A
%   \def\opt#1{#1}
% }
%
%
% ^^A Have the Index at 'section' level rather than 'part'.  Otherwise it is
% ^^A just the same definition from 'l3doc.cls'.
% \IndexPrologue{^^A
%   \section*{Index}
%   \markboth{Index}{Index}
%   \addcontentsline{toc}{section}{Index}
%   The italic numbers denote the pages where the corresponding entry is
%   described, numbers underlined point to the definition, all others indicate
%   the places where it is used.^^A
% }
%
%
% \GetFileInfo{postnotes.sty}
%
% \title{^^A
%   The \pkg{postnotes} package^^A
%   \thanks{This file describes \fileversion, released \filedate.}^^A
%   \texorpdfstring{\\{}\medskip{}}{ - }^^A
%   Code documentation^^A
%   \texorpdfstring{\medskip{}}{}^^A
% }
%
% \author{^^A
%   Gustavo Barros^^A
%   \thanks{\url{https://github.com/gusbrs/postnotes}}^^A
% }
%
% \date{\filedate}
%
% \maketitle
%
%
% \tableofcontents
%
%
% \section{Initial setup}
% Start the \pkg{DocStrip} guards.
%    \begin{macrocode}
%<*package>
%    \end{macrocode}
%
% Identify the internal prefix (\LaTeX3 \pkg{DocStrip} convention).
%    \begin{macrocode}
%<@@=postnotes>
%    \end{macrocode}
%
%
%    \begin{macrocode}
\ProvidesExplPackage {postnotes} {2022-03-24} {0.1.0}
  {Endnotes for LaTeX}
%    \end{macrocode}
%
%
% \section{Data}
%
%    \begin{macrocode}
\int_new:N \g_@@_id_int
\tl_new:N \g_@@_theid_tl
\tl_gset:Nn \g_@@_theid_tl { \int_use:N \g_@@_id_int }
\seq_new:N \g_@@_queue_seq

% <note theid>
\cs_new:Npn \@@_data_name:n #1
  { g_@@_ #1 _data_prop }
\cs_generate_variant:Nn \@@_data_name:n { e }

% <note theid><note content>
\cs_new_protected:Npn \@@_store:nn #1#2
  {
    \prop_new:c { \@@_data_name:e {#1} }
    \prop_gput:cnn { \@@_data_name:e {#1} } { type } { note }
    \prop_gput:cnx { \@@_data_name:e {#1} } { mark } { \l_@@_mark_tl }
    \prop_gput:cnx { \@@_data_name:e {#1} } { sort }
      {
        \bool_if:NTF \l_@@_manual_sort_bool
          { \fp_use:N \l_@@_sort_fp }
          { \int_use:N \c@postnote }
      }
    \prop_gput:cnx { \@@_data_name:e {#1} } { thechapter } { \thechapter }
    \prop_gput:cnx { \@@_data_name:e {#1} } { thesection } { \thesection }
    \prop_gput:cnx { \@@_data_name:e {#1} }
      { pnsectid } { \int_use:N \g_@@_sectid_int }
    \prop_gput:cnn { \@@_data_name:e {#1} } { content } {#2}
    \prop_show:c { \@@_data_name:e {#1} }
  }

% <note theid><note content>
\cs_new_protected:Npn \@@_section_store:nn #1#2
  {
    \prop_new:c { \@@_data_name:e {#1} }
    \prop_gput:cnn { \@@_data_name:e {#1} } { type } { section }
    \prop_gput:cnx { \@@_data_name:e {#1} } { thechapter } { \thechapter }
    \prop_gput:cnx { \@@_data_name:e {#1} } { thesection } { \thesection }
    \prop_gput:cnx { \@@_data_name:e {#1} }
      { pnsectid } { \int_use:N \g_@@_sectid_int }
    \prop_gput:cnn { \@@_data_name:e {#1} } { content } {#2}
  }


\int_new:N \g_@@_abspage_int
\AddToHook { shipout/before } { \int_gincr:N \g_@@_abspage_int }

\tl_const:Nn \c_@@_ref_prefix_tl { postnote@r }
\cs_new_protected:Npx \post@note
  { \exp_not:N \@newl@bel { \c_@@_ref_prefix_tl } }

\cs_new:Npn \@@_set_mark_label:n #1
  { \iow_shipout_x:Nn \@auxout { \post@note { mark@ #1} { \thepage } } }
\cs_generate_variant:Nn \@@_set_mark_label:n {x}
\cs_new:Npn \@@_set_text_label:n #1
  {
    \iow_shipout_x:Nn \@auxout
      { \post@note { text@ #1} { \int_use:N \g_@@_abspage_int } }
  }

\cs_new:Npn \@@_get_text_pageref:n #1
  {
    \cs_if_exist:cTF { \c_@@_ref_prefix_tl @ #1 }
      { \exp_not:v { \c_@@_ref_prefix_tl @ #1 } }
      { \c_max_int }
  }
\cs_generate_variant:Nn \@@_get_text_pageref:n {e}
%    \end{macrocode}
%
%
% \section{Options}
%
%    \begin{macrocode}
\tl_new:N \l_@@_list_env_tl
\tl_set:Nn \l_@@_list_env_tl { enumerate }

\tl_new:N \l_@@_print_format_tl
\tl_set:Nn \l_@@_print_format_tl { \small }
%    \end{macrocode}
%
%
% \section{\cs{postnote}}
%
% Different from the traditional \cs{footnotemark} / \cs{footnotetext} system,
% in the context of endnotes, the functionality which corresponds to
% \cs{footnotetext} is simply to store the data to be typeset later.  Hence,
% some of the problems that afflict footnotes do not apply to endnotes.
% Namely, they can be used in ``inner paragraph mode'' (\cs{mbox} etc.), and
% if the ``text'' will be typeset in the same page as the ``mark'' is of
% little concern.
%
% Therefore, the separation between ``mark'' and ``text'' is of little use for
% endnotes and, that given, the ability to set an arbitrary ``mark'' value is
% less important, even if it has a place.  And, since they introduce their own
% set of problems, and quite ponderous ones, we are better off curbing some of
% this heritage.
%
% One relevant issue that potentially requires some manual adjustment, just as
% they do for \cs{footnotes}, are floats and their ability to wreak havoc in
% an otherwise so well behaved order of counters.  If we received an arbitrary
% number, as the kernel functionality for footnotes and other endnotes
% packages do, this value is expected to the printed as such, hence it must
% correspond to the \texttt{postnote} counter.  But this counter is in the
% hands of the user, and can be reset along the document, thus its uniqueness
% cannot be ensured.  Besides, there might be a need for a mark to be an
% arbitrary token list, rather than a number.  For example, a critical edition
% where the original order of notes cannot be disturbed, and editor notes have
% to be intercalated among the original ones.  The kernel footnote structure
% does not support this (the optional argument must be a number), but it would
% be interesting if we could.  The best idea I came up with is to enjoy
% ``modernity'' and leverage a keyval set of options: \opt{mark}, receiving an
% arbitrary token list, and \opt{sort}, receiving a floating point number.
% The pair makes for well defined mark and print order.  And it also covers
% the case of floats, even if they might also need an ocasional
% \texttt{\cs{addtocounter}\{postnote\}}.
%
% We don't offer \cs{footnotetext} at all.  But, we offer a stripped down
% version of \cs{footnotemark}, to enable users to have multiple marks for the
% same note.  It receives only an optional argument which is a note offset,
% and can only receive non-negative values and refer to previously added
% notes.
%
% David Carlisle and Ulrike Fischer shared some thoughts on the matter on the
% TeX.SX chat:
% \url{https://chat.stackexchange.com/transcript/message/60754383#60754383}.
%
%

%    \begin{macrocode}
\newcounter { postnote }

\NewDocumentCommand \postnote { O { } m }
  { \@@_note:nn {#1} {#2} }

% <options (currently just the mark)><note content>
\cs_new_protected:Npn \@@_note:nn #1#2
  {
    \group_begin:
    \keys_set:nn { postnotes/note } {#1}
    \int_gincr:N \g_@@_id_int
    \seq_gput_right:Nx \g_@@_queue_seq { \g_@@_theid_tl }
    \tl_if_empty:NT \l_@@_mark_tl
      {
        \stepcounter { postnote }
        \tl_set:Nx \l_@@_mark_tl { \thepostnote }
      }
    \@@_set_mark_label:x { \g_@@_theid_tl }
    \@@_typeset_mark:xV { \g_@@_theid_tl } \l_@@_mark_tl
    \@@_store:nn { \g_@@_theid_tl } {#2}
    \group_end:
  }

\tl_new:N \l_@@_mark_tl
\fp_new:N \l_@@_sort_fp
\bool_new:N \l_@@_manual_sort_bool
\keys_define:nn { postnotes/note }
  {
    mark .tl_set:N = \l_@@_mark_tl ,
    mark .value_required:n = true ,
    sort .code:n =
      {
        \fp_set:Nn \l_@@_sort_fp {#1}
        \bool_set_true:N \l_@@_manual_sort_bool
      } ,
    sort .value_required:n = true ,
  }

\tl_new:N \l_@@_saved_spacefactor_tl
\bool_new:N \l_@@_manual_mark_bool
% <theid><mark>
\cs_new_protected:Npn \@@_typeset_mark:nn #1#2
  {
    \mode_leave_vertical:
    \if_mode_horizontal:
    \tl_set:Nx \l_@@_saved_spacefactor_tl { \the\spacefactor }
    \nobreak
    \fi:
    \bool_if:NTF \g_@@_hyperref_bool
      {
        \bool_if:NF \l_@@_manual_mark_bool
          {
            \Hy@raisedlink
              { \hyper@anchorstart { postnote. #1 .mark } \hyper@anchorend }
          }
        \hyperlink { postnote. #1 .text } { \@@_mark_format:n {#2} }
      }
      { \@@_mark_format:n {#2} }
    \if_mode_horizontal:
    \spacefactor \l_@@_saved_spacefactor_tl
    \fi:
    \scan_stop:
  }
\cs_generate_variant:Nn \@@_typeset_mark:nn { xV }
\cs_new_protected:Npn \@@_mark_format:n #1
  { \hbox:n { \@textsuperscript { \normalfont #1 } } }


\NewDocumentCommand \postnotemark { O { 0 } }
  { \@@_mark:n {#1}  }
% <number>
\cs_new_protected:Npn \@@_mark:n #1
  {
    \group_begin:
    \bool_set_true:N \l_@@_manual_mark_bool
    \int_compare:nNnTF {#1} < { 0 }
      {
        % TODO warn: must be positive, using 0
        \int_zero:N \l_tmpa_int
      }
      { \int_set:Nn \l_tmpa_int {#1} }
    \int_compare:nNnT { \g_@@_id_int } > { 0 }
      {
        \int_set_eq:NN \l_tmpb_int \g_@@_id_int
        \int_until_do:nNnn { \l_tmpa_int } = { 0 }
          {
            \int_decr:N \l_tmpb_int
            \@@_prop_get:nnN { \int_use:N \l_tmpb_int } { type } \l_tmpa_tl
            \tl_if_eq:NnT \l_tmpa_tl { note }
              {
                \int_decr:N \l_tmpa_int
                \tl_set:Nx \l_tmpb_tl { \int_use:N \l_tmpb_int }
              }
            \int_compare:nNnT { \l_tmpb_int } < { 1 }
              {
                \int_zero:N \l_tmpa_int
                \int_set:Nn \l_tmpb_int { \l_tmpb_tl }
                % TODO warn: not enough notes for this number, using last
                % available.
              }
          }
      }
    \@@_prop_get:nnN { \int_use:N \l_tmpb_int } { mark } \l_@@_mark_tl
    \@@_typeset_mark:xV { \int_use:N \l_tmpb_int } \l_@@_mark_tl
    \group_end:
  }
%    \end{macrocode}
%
%
% \section{\cs{postnotesection}}
%
%    \begin{macrocode}
\int_new:N \g_@@_sectid_int

\NewDocumentCommand \postnotesection { O { } m }
  { \@@_section:nn {#1} {#2} }

\cs_new_protected:Npn \@@_section:nn #1#2
  {
    \int_gincr:N \g_@@_id_int
    \seq_gput_right:Nx \g_@@_queue_seq { \g_@@_theid_tl }
    \int_gincr:N \g_@@_sectid_int
    \@@_section_store:nn { \g_@@_theid_tl } {#2}
  }
%    \end{macrocode}
%
%
% \section{\cs{printpostnotes}}
%
%    \begin{macrocode}
\NewDocumentCommand \printpostnotes { O { } }
  { \@@_print_notes:n {#1} }

\tl_new:N \l_@@_print_theid_tl
\tl_new:N \l_@@_print_mark_tl
\tl_new:N \l_@@_print_sectid_tl
\tl_new:N \l_@@_curr_type_tl
\tl_new:N \l_@@_next_type_tl
\tl_new:N \l_@@_content_tl
\tl_new:N \pnthechapter
\tl_new:N \pnthesection

\newcounter { postnote@text }
\newcounter { postnote@sectid }

% <theid><prop><tl var>
\cs_new_protected:Npn \@@_prop_get:nnN #1#2#3
  {
    \prop_get:cnNF { \@@_data_name:e {#1} } {#2} #3
      { \tl_clear:N #3 }
  }

\cs_new_protected:Npn \@@_print_notes:n #1
  {
    \group_begin:
    \seq_if_empty:NF \g_@@_queue_seq
      {
        \chapter{Notes}
        % Keep biblatex happy.
        \bool_if:NT \g_@@_biblatex_bool
          { \toggletrue { blx@footnote } }
        \seq_gsort:Nn \g_@@_queue_seq
          {
            \@@_prop_get:nnN {##1} { pnsectid } \l_tmpa_tl
            \@@_prop_get:nnN {##2} { pnsectid } \l_tmpb_tl
            \tl_if_eq:NNTF \l_tmpa_tl \l_tmpb_tl
              {
                \@@_prop_get:nnN {##1} { type } \l_tmpa_tl
                \@@_prop_get:nnN {##2} { type } \l_tmpb_tl
                \bool_lazy_and:nnTF
                  { \str_if_eq_p:Vn \l_tmpa_tl { note } }
                  { \str_if_eq_p:Vn \l_tmpb_tl { note } }
                  {
                    \@@_prop_get:nnN {##1} { sort } \l_tmpa_tl
                    \@@_prop_get:nnN {##2} { sort } \l_tmpb_tl
                    \fp_compare:nNnTF { \l_tmpa_tl } > { \l_tmpb_tl }
                      { \sort_return_swapped: }
                      { \sort_return_same:    }
                  }
                  { \sort_return_same: }
              }
              { \sort_return_same: }
          }
      }
    \bool_until_do:nn { \seq_if_empty_p:N \g_@@_queue_seq }
      {
        \seq_gpop_left:NN \g_@@_queue_seq \l_@@_print_theid_tl
        \@@_prop_get:nnN { \l_@@_print_theid_tl }
          { type } \l_@@_curr_type_tl
        \int_compare:nNnTF { \l_@@_print_theid_tl } = { \g_@@_id_int }
          { \tl_set:Nn \l_@@_next_type_tl { close } }
          {
            \@@_prop_get:nnN
              { \int_eval:n { \l_@@_print_theid_tl + 1 } }
              { type } \l_@@_next_type_tl
          }
        \tl_if_eq:NnTF \l_@@_curr_type_tl { section }
          {
            \tl_if_eq:NnT \l_@@_next_type_tl { note }
              {
                \group_begin:
                \@@_prop_get:nnN { \l_@@_print_theid_tl }
                  { thechapter } \pnthechapter
                \@@_prop_get:nnN { \l_@@_print_theid_tl }
                  { thesection } \pnthesection
                \@@_prop_get:nnN { \l_@@_print_theid_tl }
                  { content } \l_@@_content_tl
                \l_@@_content_tl
                \group_end:
                \@@_prop_get:nnN { \l_@@_print_theid_tl }
                  { pnsectid } \l_@@_print_sectid_tl
                \exp_args:Nnx \setcounter
                  { postnote@sectid } { \int_eval:n { \l_@@_print_sectid_tl } }
                \exp_args:Nx \begin { \l_@@_list_env_tl }
                \l_@@_print_format_tl
              }
          }
          { % curr_type = 'note'
            \tl_if_eq:NNF \@currenvir \l_@@_list_env_tl
              {
                \exp_args:Nx \begin { \l_@@_list_env_tl }
                \l_@@_print_format_tl
              }
            \@@_prop_get:nnN { \l_@@_print_theid_tl }
              { mark } \l_@@_print_mark_tl
            \item
              [
                \@@_print_mark:eV
                  { \l_@@_print_theid_tl }
                  \l_@@_print_mark_tl
              ]
            \group_begin:
            \refstepcounter { postnote@text }
            \cs_set:Npn \@currentcounter { postnote }
            \cs_set:Npx \@currentlabel
              { \p@postnote \l_@@_print_mark_tl }
            \@@_prop_get:nnN { \l_@@_print_theid_tl }
              { content } \l_@@_content_tl
            \l_@@_content_tl
            \group_end:
            \tl_if_eq:NnF \l_@@_next_type_tl { note }
              { \exp_args:Nx \end { \l_@@_list_env_tl } }
          }
      }
    \group_end:
  }

\cs_new:Npn \@@_print_mark_format:n #1 { #1 . }
\cs_new:Npn \@@_print_mark:nn #1#2
  {
    \bool_if:NTF \g_@@_hyperref_bool
      {
        \Hy@raisedlink { \hyper@anchorstart { postnote. #1 .text } \hyper@anchorend }
        \hyperlink { postnote. #1 .mark } { \@@_print_mark_format:n {#2} }
      }
      { \@@_print_mark_format:n {#2} }
    \@@_set_text_label:n {#1}
  }
\cs_generate_variant:Nn \@@_print_mark:nn {eV}
%    \end{macrocode}
%
%
% \section{Compatibility}
%
%
%    \begin{macrocode}
\bool_new:N \g_@@_hyperref_bool
\AddToHook{begindocument}
  {
    \IfPackageLoadedTF { hyperref }
      {
        \bool_gset_true:N \g_@@_hyperref_bool
        \cs_set:Npn \theHpostnote { \int_use:N \g_@@_id_int }
      }
      {}
  }
\bool_new:N \g_@@_biblatex_bool
\AddToHook{begindocument}
  {
    \IfPackageLoadedTF { biblatex }
      { \bool_gset_true:N \g_@@_biblatex_bool }
      {}
  }
%    \end{macrocode}
%
%
%    \begin{macrocode}
%</package>
%    \end{macrocode}
%
%
% \PrintIndex
%
%
